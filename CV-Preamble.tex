%%%%%%%%%%%%%%%%%%%%%%%%%%%%%%%%%%%%%%%%%%%%%%%%%%%%%%%%%%%%%%%%%
%% SIMPLE-RESUME-CV
%% <https://github.com/zachscrivena/simple-resume-cv>
%% This is free and unencumbered software released into the
%% public domain; see <http://unlicense.org> for details.
%%%%%%%%%%%%%%%%%%%%%%%%%%%%%%%%%%%%%%%%%%%%%%%%%%%%%%%%%%%%%%%%%

% Long table for page layout.
\usepackage{longtable}

% Geometry package for page margins.
\usepackage[
left=0.75in,
right=0.75in,
top=0.75in,
bottom=0.6in,
nohead,
includefoot]{geometry}

% Hyphenation: Disabled.
\usepackage[none]{hyphenat}

% XeLaTeX packages.
\usepackage{fontspec}
\defaultfontfeatures{Ligatures=TeX}
\usepackage{xunicode}
\usepackage{xltxtra}

% Font: Use "Tinos" as the main typeface (\textnormal{}, \normalfont).
% The "Tinos" fonts are released under the Apache License Version 2.0,
% and can be downloaded for free at <http://www.fontsquirrel.com/fonts/tinos>.
% Symbol table: <http://www.fileformat.info/info/unicode/font/tinos/grid.htm>
\setmainfont
[Path=./Fonts/,
ItalicFont=Tinos-Italic,
BoldFont=Tinos-Bold,
BoldItalicFont=Tinos-BoldItalic]
{Tinos-Regular.ttf}

% Sans-serif font: Switched to "Tinos".
\renewcommand{\sffamily}{\rmfamily}

% Typewriter (monospace) font: Switched to "Tinos".
\renewcommand{\ttfamily}{\rmfamily}

% Small caps font: Switched to "Tinos".
\renewcommand{\scshape}{\rmfamily}

% Symbols (unicode).
\newcommand{\bulletsymbol}{\char"25CF}
\newcommand{\tildesymbol}{\char"007E}

% PDF settings and properties.
\usepackage{hyperref}

% Headers and footers: Blank header, page number in footer.
\makeatletter
\def\ps@plain{%
\def\@oddhead{}%
\def\@evenhead{}%
\def\@oddfoot{\footnotesize\hfill{\thepage}\hfill}%
\def\@evenfoot{\footnotesize\hfill{\thepage}\hfill}}
\makeatother

\pagestyle{plain}

% Paragraph style: No indentation.
\setlength{\parindent}{0in}

% Footnotes: Use symbols instead of numbers for labels.
\renewcommand{\thefootnote}{\fnsymbol{footnote}}

% Abbreviations for months.
\newcommand{\longmonth}[1]{%
\ifcase#1\relax
\or January%
\or February%
\or March%
\or April%
\or May%
\or June%
\or July%
\or August%
\or September%
\or October%
\or November%
\or December%
\fi}
\newcommand{\shortmonth}[1]{%
\ifcase#1\relax
\or Jan%
\or Feb%
\or Mar%
\or Apr%
\or May%
\or Jun%
\or Jul%
\or Aug%
\or Sep%
\or Oct%
\or Nov%
\or Dec%
\fi}

% Select datestamp format.
\def\DatestampFormatSelection{2}

% Datestamp format: {yyyy}{MM}{dd} ---> yyyy-MM-dd (e.g., 2010-12-31).
\ifnum\DatestampFormatSelection=1
\newcommand{\datestampYMD}[3]{\mbox{#1-#2-#3}}
\newcommand{\datestampYM}[2]{\mbox{#1-#2}}
\newcommand{\datestampY}[1]{#1}
\fi

% Datestamp format: {yyyy}{MM}{dd} ---> MMM yyyy (e.g., Dec 2010).
\ifnum\DatestampFormatSelection=2
\newcommand{\datestampYMD}[3]{\mbox{\shortmonth{#2} #1}}
\newcommand{\datestampYM}[2]{\mbox{\shortmonth{#2} #1}}
\newcommand{\datestampY}[1]{#1}
\fi

% Datestamp format: {yyyy}{MM}{dd} ---> MMMM yyyy (e.g., December 2010).
\ifnum\DatestampFormatSelection=3
\newcommand{\datestampYMD}[3]{\mbox{\longmonth{#2} #1}}
\newcommand{\datestampYM}[2]{\mbox{\longmonth{#2} #1}}
\newcommand{\datestampY}[1]{#1}
\fi

% Datestamp format: {yyyy}{MM}{dd} ---> yyyy (e.g., 2010).
\ifnum\DatestampFormatSelection=4
\newcommand{\datestampYMD}[3]{#1}
\newcommand{\datestampYM}[2]{#1}
\newcommand{\datestampY}[1]{#1}
\fi

% Macro: title (name).
\renewcommand{\title}[1]{%
\pdfbookmark{#1}{#1}%
\begin{center}%
\begin{Huge}%
\textbf{#1}%
\end{Huge}%
\end{center}%
\vspace{-1.75em}}

% Macro: subtitle (personal information below name).
\newenvironment{subtitle}
{\par\begin{center}%
\begin{footnotesize}}
{\par\end{footnotesize}%
\end{center}}

% Macro: body (rest of the document).
\newenvironment{body}
{\vspace{-2em}
\begin{longtable}{p{0.15\textwidth}p{0.80\textwidth}}}
{\end{longtable}}

% Macro: section (new section for Education, Research Experience, etc.).
\renewcommand{\section}[2]{\vspace{-1em}\\\pdfbookmark{#1}{#1}\\%
\fontsize{9pt}{11pt}\selectfont\raggedright\textbf{\MakeUppercase{#2}}&}

% Macro: nextentry (new entry within the same section).
\def\nextentry{\vspace{-0.5em}\\~&}

% Macro: entrygap (small vertical gap within a long entry).
\def\entrygap{\vspace{0.3em}}

% Macro: bibgap (small horizontal gap for bibliographic entries).
\def\bibgap{\hspace{0.5em}\ignorespaces}

% Macro: detail (text in smaller font under an entry).
\newenvironment{detail}
{\par\begin{small}}
{\par\end{small}}

% Macro: hide.
\newcommand{\hide}[1]{}
